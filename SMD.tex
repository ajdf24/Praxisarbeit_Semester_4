\section{Der SMD-Assistent im GDS-System}
Die \ac{SMD} enthalten alle wichtigen Informationen um die Strucktur einer Klasse zu beschreiben. Diese Klasseninformationen sollen sp\"ater von einem SMD-Assistent verwaltet werden.

In einer MySQL-Datenbank k\"onnen alle Klasseninformationen abgelegt werden. Zu diesen Informationen z\"ahlen zum Beispiel der Klassenname, Attribute mit Modifier, Methoden mit Modifier Parametern und R\"uckgabetyp. 

Da es sich bei den \ac{SMD} um reine Metadaten handelt, werden keine Methodenr\"umpfe in die \ac{SMD}-Datenbank aufgenommen.

Der Nutzer des \ac{GDS}-Systems gibt \"uber den \ac{UDDE}, also das User Interface, seine Klassen und \ac{AMD} in das System.
Unter \ac{AMD} werden die zu seinen Klassen passenden Instanzen bezeichnet, welche der User der Anwendung zum Speichern \"ubergibt.

Der SMD-Assistent ist also f\"ur das Umwandeln von Klassen in \ac{SMD}s verantwortlich. Nach dem erstellen der \ac{SMD}s werden diese zur Aufbewahrung vom Assistenten dem \ac{GDS} \"ubergeben, welcher die Ablage der Daten in einer Datenbank verwaltet.

Die \ac{SMD}s werden von der Anwendung an verschiedenen Stellen ben\"otigt. Der \ac{CG} wandelt die \ac{SMD}s wieder in Java-Klassen, welche OPM-Konform erzeugt werden. Bei den erstellten Klassen handelt es sich nat\"urlich nur um einfache Ger\"uste, aber mehr wird f\"ur eine Serialisierung beziehungsweise Deserialisierung nicht ben\"otigt.
Die generierten Datenstr\"ome werden dann ebenfalls vom \ac{GDS} in einer Datenbank gespeichert.

Der \ac{IG} erstellt aus den vom SMD-Assistenten gelieferten \ac{SMD}s Schemen f\"ur JSON und XML. Eine Funktion zur Erstellung des JSON-Schemas aus einer Klasse wird in Kapitel \ref{JSON-Schema} beschreiben.


