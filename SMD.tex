\section{Der SMD-Assistent im GDS-System} \label{SMD-Assistent}
Die \ac{SMD} enthalten alle wichtigen Informationen um die Strucktur einer Klasse zu beschreiben. Diese Klasseninformationen sollen sp\"ater von einem SMD-Assistenten verwaltet werden.

In einer MySQL-Datenbank k\"onnen alle Klasseninformationen abgelegt werden. Zu diesen Informationen z\"ahlen zum Beispiel der Klassenname, Attribute mit Modifier, Methoden mit Modifier Parametern und R\"uckgabetyp. 

Da es sich bei den \ac{SMD} um reine Metadaten handelt, werden keine Methodenr\"umpfe in die \ac{SMD}-Datenbank aufgenommen.

Der Nutzer des \ac{GDS}-Systems gibt \"uber den \ac{UDDE}, also das User Interface, seine Klassen und \ac{AMD} in das System. Er kann \"uber den \ac{UDDE} ein komplettes Klassengr\"ust erstellen, indem er alle Klassenattribute und Methoden angibt. 

Hierbei ist er nat\"urlich auch in der Lage den Modifier und den R\"uckgabetyp zu bestimmen. Bei Methoden k\"onnen nat\"urlich zus\"atzlich Parameterlisten angegeben werden.
% Unter \ac{AMD} werden die zu seinen Klassen passenden Instanzen bezeichnet, welche der User der Anwendung zum Speichern \"ubergibt.

Des weiteren legt der \ac{UDDE} die generierten "`strukturellen Metadaten"' in einer Datenbank ab, welche sp\"ater vom SMD-Assistenten ausgelesen werden k\"onnen.

Der SMD-Assistent ist f\"ur das Umwandeln von Klassen in \ac{SMD} verantwortlich. Nach dem erstellen der \ac{SMD} werden diese zur Aufbewahrung vom Assistenten in einer Datenbank verwaltet.

Die \ac{SMD} werden von der Anwendung an verschiedenen Stellen ben\"otigt. Ein \ac{CG} wandelt die \ac{SMD} wieder in Programmcode, welcher OPM-Konform erzeugt wird. Eine genaue Spezifikation f\"ur den Klassengenerator ist im Kapitel \ref{CG} zu finden.
Die generierten Daten k\"onnen dann ebenfalls vom \ac{GDS} in einer Datenbank gespeichert werden.

Der \ac{IG} erstellt aus den vom SMD-Assistenten gelieferten \ac{SMD}s Schemen f\"ur JSON und XML. Eine Funktion zur Erstellung des JSON-Schemas aus einer Klasse wird in Kapitel \ref{JSON-Schema} und fortfolgenden beschreiben.
Auch das entsprechnde Klassenschema soll vom \ac{GDS} in einer Datenbank abgelegt werden.

Der beschriebene Zusammenhang der Komponenten des \ac{GDS} ist im Bild unten noch einmal verdeutlicht.

\begin{figure}[!ht]
\centering
\includegraphics[width=13.5cm]{Bilder/UebersichtGDS}
\label{GDS \"Ubersicht}
\caption{GDS \"Ubersicht}
\centering
\end{figure}

\FloatBarrier

\subsection{Spezifikation des SMD-Assistenten}
Im Verlauf der Arbeit wurde zunehmend klar, das die Spezifikation des SMD-Assistent n\"otig wird. Da die \ac{SMD}, wie im Schaubild oben zu erkennen, zentraler Bestandteil des Projektes sind.
Der SMD-Assistent wurde im Projekt \"offter diskutiert und soll an dieser Stelle einmal genauer erleutert werden.

\subsubsection{Funktionen des SMD-Assistenten}
Der SMD-Assistent soll die "`strukturellen Metadaten"' aus der Datenbank laden und diese an das \ac{GDS}, den \ac{IG} oder den \ac{CG} weiterreichen. 

Zu einer ObjectID, einer ClassID oder einem Klassennamen m\"ussen die passenden Metadaten aus der Datenbank geladen werden.
Die geladenen Daten werden in einer Instanz der Klasse \texttt{ClassDecr} zusammengafasst und mittels dieser Instanz \"ubergeben.

Objekte, welche serialisiert werden k\"onnen, haben eine ObjektID, anhand welcher sie eindeutig identifiziert werden k\"onnen. Zus\"atzlich k\"onnen \"uber die ObjectID auch verschiedene Instanzen eines Objektes unterschieden und zugeordnet werden.

Mittels einer ClassID k\"onnen genau wie mit der ObjectID Objekte identifiziert werden, jedoch keine Instanzen.

Eine Instanz der Klasse \texttt{ClassDecr} enth\"alt mittels der Klassen \texttt{MethodDescr} und \texttt{AttrDecr} alle n\"otigen Informationen um eine Klasse rekonstruieren zu k\"onnen.

\texttt{MethodDescr} enth\"alt eine Methode der Klasse, mit einer Liste von allen Parametern und deren Typen.
Die Klasse \texttt{AttrDecr} hingegen h\"alt jeweils ein Attribut mit dessen Informationen wie zum Beispiel Type und Modifier. \cite{Zil14}

Der Zusammenhang ist im Klassendiagramm der SMD-Klassen noch einmal dargestellt. Im Bild 3 ist aus Gr\"unden der \"Ubersichtlichkeit nicht der gesamte OPM-Klassenbaum abgebildet, sondern lediglich die f\"ur den SMD-Assistent wichtige Klassen sind aufgef\"uhrt.

Einmal aus der Datenbank geladene \ac{SMD}s soll der SMD-Assistent zwischenspeichern, um beim erneuten Abfragen schneller reagieren zu k\"onnen. 

Damit es nicht zu einem Arbeitsspeicher\"uberlauf kommt, muss der SMD-Assistent genau wie zuk\"unftige andere Assistenten auch eine M\"oglichkeit besitzen seinen internen Cache zu verkleinern. Dies soll \"uber einen m\"oglichst effizienten Scheduling-Algorithmus geschehen, welcher implementiert und auf Effizienz gepr\"uft werden muss.

Hier kommt ein weiterer Assistent zum Einsatz, und zwar der Speicher-Assistent, welcher bei geringem Arbeitsspeicher einen Befehl an alle Assistenten schickt, damit diese ihren Speicherbedarf reduzieren. Durch die Serialisierer wurde die \"Uberlegung zum Speicher-Assistenten zum ersten mal entfacht, da hier teilweise sehr viel Arbeitsspeicher ben\"otigt wird. Dazu aber im n\"achsten Kapitel mehr.

\begin{figure}[!ht]
\centering
\includegraphics[width=13cm]{Bilder/SMD_Klassen}
\title{Klassendiagramm der SMD-Klassen}
\caption{Klassendiagramm der SMD-Klassen}
\centering
\end{figure}

\subsubsection{Der SMD-Assistent unter Java}
Unter Java kann die Arbeitsweise des SMD-Assistent deutlich vereinfacht werden, da hier die \ac{SMD} nur bei explizitem Verlangen geliefert werden m\"ussen, denn Java verwendet mit den \textit{.class}-Objekten eigene Metadaten \"uber die eine Verarbeitung unter Java einfacher Umgesetzt werden kann. 
Der SMD-Assistent sollte unter Java also in der Lage sein direkt Klassenobjekte zu liefern.

\subsection{Der Klassengenerator} \label{CG}
Der Klassengenerator ist ein Modul des \ac{GDS}, welcher aus den vom SMD-Assistenten gegebenen strukturellen Metadaten wieder Programmcode generiert. Es gibt eine abstrakte Klasse \texttt{ClassGenerator} welche abstrakte Methoden zur Verf\"ugung stellt. 

F\"ur jede Programmiersprache, soll nun ein Klassengenerator von der abstrakten Oberklasse abgeleitet werden. Die Programmiersprachen spezifischen Klassengeneratoren sind dann in der Lage, f\"ur ihre Programmiersprache aus den \ac{SMD} Klassen zu generieren.

Da die Serialisierer wie Jackson und JAXB f\"ur Attribute, die nicht \texttt{public} sind, Getter- und Setter-Methoden ben\"otigen muss der Klassengenerator diese erzeugen, auch wenn diese nicht in den \ac{SMD} auftauchen. Des weiteren muss er diese Methoden auch nach dem OPM-Standard implementieren. Dies bedeutet, das in Setter-Methoden die Attribute gesetzt und in Getter-Methoden die Attribute gelesen werden m\"ussen.

Au\ss{}erdem ben\"otigen die Serialisierer auch Standardkonstruktoren um ein leeres Element zu erstellen, welches im Laufe der Deserialisierung gef\"ullt werden kann.

Um die Serialisierer steuern zu k\"onnen sind gegebenenfalls Annotationen an den Klassen notwendig, welche ebenfalls vom \texttt{ClassGenerator} angebracht werden m\"ussen.



