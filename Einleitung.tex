\section{Einleitung}
Die folgende Arbeit befasst sich mit der Serialisierung und Deserialisierung von Datenobjekten in \ac{JSON}-Format zur �bertragung von Objektdaten, die sogenannten Anwendungsdaten, welche in Programmen generiert werden.
\subsection{Generisches Managementsystem f\"ur Energiedaten}
\ac{GDS} ist das generisches Managementsystem f\"ur Energiedaten. Es verwendet Objekte um Anwendungsdaten zu verwalten, die einem \ac{OPM} gen�gen und durch \ac{SMD} beschrieben werden k\"onnen.

Alle Programmbestandteile des \ac{GDS} sind durch das Objektorientierte Datenmodell programmiersprachenunabh�ngig und k�nnen durch Struckturelle Metadaten beschrieben werden.

Das \ac{GDS} wird zur Zeit im \ac{IAI} des \ac{KIT} entwickelt und ist ein System generischer Datenservices welches bei Fertigstellung vollautomatisch Energiedaten managen soll. 

Durch den generischen Charakter kann es aber auch Daten aus anderen Bereichen verwalten, wenn diese dem \ac{OPM}-Modell gen\"ugen.
\subsection{Objektorientiertes Datenmodell}
Im \ac{OPM}  werden Richtlinien f�r die Entwicklung von allen objektorientierten Softwarebausteinen festgelegt. Das gesamte Projekt ist bisher OPM-konform gehalten und somit soll auch die Schnittstelle der Serialisierung OPM-konform gestaltet werden.

\ac{OPM} ist im weiterem Sinn eine Abstraktionsschicht, mit dessen Hilfe ein Programm programmiersprachenunabh�ngig, beschrieben werden kann.

Um diese Unabh�ngigkeit zu erreichen, umfasst das Modell viele Regeln, wie Programmcode aussehen soll und welchen Anforerungen er gen�gen muss.

Im Folgenden sind die wichtigsten OPM-Regeln dargstellt:
\begin{itemize}
 \item Basisklasse \texttt{OPMObject} von der alle Klassen erben
 \item Es gibt keine Konstanten
 \item Attribute sind grunds\"atzlich \texttt{private}, also von auserhalb der Klasse nicht \"anderbar, und werden gegebenenfalls durch Getter- und Setter-Methoden aufgerufen (\texttt{private} also nicht im Sinn von Java)
 \item Programme bestehen nur aus Objekten, der Aufruf erfolgt ausschlie\ss{}lich \"uber Methodenaufrufe
 \item Es k\"onnen Standarddatentypen der jeweiligen Programmiersprache verwendet werden
 \item Methodenbezeichner werden im "`Camel Case"' formuliert
 \item Attributbezeichner werden im "`Lower Camel Case"' formuliert
 \item In der Dokumentation eines Attributs wird immer der erlaubte Wertebereich spezifiziert
 \item Die Dokumentation der Programmcodes muss den Spezifikationen der verwendete Sprache entsprechen, damit diese auch als Dokumentation vom Compiler erkannt wird
 \item Jede Klasse wird mit einem Status Valid, Experimental oder Depricated in ihrer Dokumentation beschrieben, um den Entwicklungsstand sofort erkennen zu k�nnen
\end{itemize}
Alle Klassen die vom \ac{GDS} verwaltet werden sollen, m\"ussen diesen Grunds\"atzen gen\"ugen um verarbeitet werden zu k\"onnen.

Des weiteren gibt es, wie eben schon kurz dargestellt, im OPM-Modell eine Oberklasse OPMObject, von der alle Klassen erben sollen. Dieses vorgehen ist zum Beispiel auch in Java umgesetzt, wo jede Klasse von \texttt{Object} erbt.

Vorteil dieses Konzepts ist, das Methoden, Kostruktoren und Attribute, welche jede Klasse haben soll nur einmal in der Oberklasse aufgef\"uhrt werden m\"ussen und dann vererbt werden.

\subsection{Strukturelle Metadaten}
Strukturelle Metadaten sind laut der \ac{OPM}-Definition spezielle Metadaten, die den Aufbau einer Programmklasse enthalten. Die Informationen der \ac{SMD} sind programmiersprachenunabh�ngig und stehen dem Anwendungsprogrammen zur Verf�gung. Die Strukturellen Metadaten sind f�r jede Klasse vorhanden und enthalten unter anderem Attributnamen, Attributtyp und den dazugeh\"origen Quallifier auserdem Methodennamen, Methodenattribute, den r\"uckgabewert und den Quallifier. Siehe Kapitel \ref{SMD-Assistent}.

Die Metadaten werden in einer SQL-Datenbank gespeichert, aus der sie, wenn ben�tigt, geladen werden k�nnen. Diese Arbeit �bernimmt ein Assistent, welcher im Kapitel \ref{SMD-Assistent} noch genauer beschrieben wird. \cite{Zil14}
