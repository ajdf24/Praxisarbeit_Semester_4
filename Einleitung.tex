\section{Einleitung}
Die folgende Arbeit befasst sich mit der Serialisierung von Datenobjekten in JSON zur �bertragung von Objektdaten. 
\subsection{Generisches Managementsystem f\"ur Energiedaten}
Das \ac{GDS} verwendet Objekte um Anwendungsdaten zu verwalten, die einem \ac{OPM} gen�gen und durch \ac{SMD} beschrieben werden k\"onnen.
Das \ac{GDS} wird zur Zeit im \ac{IAI} des \ac{KIT} entwickelt und ist ein System generischer Datenservices welches bei 
Vertrigstellung vollautomatisch Energiedaten managen soll. 

Durch den generischen Charakter kann es aber auch Daten aus anderen Bereichen verwalten, wenn diese dem \ac{OPM}-Modell gen\"ugen.
\subsection{Objektorientiertes Datenmodell}
Im \ac{OPM}  werden Richtlinien f�r die Entwicklung von allen objektorientierten Softwarebausteinen festgelegt. Das gesamte Projekt ist bisher OPM-konform gehalten und somit soll auch die Schnittstelle der Serialisierung OPM-konform gestaltet werden.

Im Folgenden sind die wichtigsten OPM-Regeln dargstellt:
\begin{itemize}
 \item Basisklasse \texttt{OPMObject} von der alle Klassen erben
 \item Es gibt keine Konstanten
 \item Attribute sind grunds\"atzlich \texttt{private}, also von auserhalb der Klasse nicht \"anderbar, und werden gegebenenfalls durch Getter- und Setter-Methoden aufgerufen
 \item Programme bestehen nur aus Objekten, der Aufruf erfolgt ausschlie\ss{}lich \"Uber Methodenaufrufe
 \item Es k\"onnen Standarddatentypen verwendet werden
 \item Methodenbezeichner werden im "`Camel Case"' formuliert
 \item Attributbezeichner werden im "`Lower Camel Case"' formuliert
 \item In der Dokumentation eines Attributs wird immer der erlaubte Wertebereich spezifiziert
 \item Die Dokumentation muss den Spezifikationen der verwendete Sprache entsprechen
%  \item Jede Klasse enth\"alt einen der Stati Valid, Experimental oder Depricated in ihrer Dokumentation
\end{itemize}
Alle Klassen die vom \ac{GDS} verwaltet werden sollen, m\"ussen diesen Grunds\"atzen gen\"ugen um verarbeitet werden zu k\"onnen.
\subsection{Strukturelle Metadaten}
Strukturelle Metadaten sind laut der \ac{OPM}-Definition spezielle Metadaten, die den Aufbau einer Programmklasse enthalten. Die Informationen der \ac{SMD} sind programmiersprachenunabh�ngig und stehen dem Anwendungsprogrammen zur Verf�gung.

Die Metadaten werden in einer SQL-Datenbank gespeichert, aus der sie, wenn ben�tigt, geladen werden k�nnen. \cite{Zil14}
