\section{Aufgabenstellung}
\label{Aufgabenstellung}
In einem \ac{GDS}-System werden Anwendungsdaten mit Hilfe von Objekten verwaltet, welche den Regeln von \ac{OPM} folgen und mittels der \ac{SMD}s beschrieben werden k�nnen.

F�r den Transfer von solchen Anwendungsdaten, sind Schnittstellen f�r die automatische Serialisierung der Daten vorgesehen, welche im Rahmen dieser und einer weiteren Arbeit entwickelt werden sollen.  \cite{Wal14} 

Ziel der Arbeit ist es, Spannungszeitreihen und OPM-Objekte mittels der Metadaten in ein JSON-Format zu Serialisieren und diese zu �bertragen.  Die Empf�ngerseite muss letztendlich in der Lage sein aus den �bertragenen Daten wieder Objekte zu erstellen.

Abschlie\ss{}end soll ein Vergleich zeigen welche der  Serialisierungsarten (JSON oder XML)  besser geeignet ist. Hierf�r wird die Arbeit "`Serialisierung von Datenobjekten in XML zur �bertragung von Objekten aus Energieanwendungen"' von Herrn Achim Walz herangezogen. \cite{Wal14}

