\section{Aufgabenstellung}
\label{Aufgabenstellung}
Im \ac{GDS}-System werden Anwendungsdaten mit Hilfe von Klassen-Objekten verwaltet, welche den Regeln von \ac{OPM} folgen und mittels der \ac{SMD} genau beschrieben werden k�nnen.

F�r den Transfer von solchen Anwendungsdaten sind Schnittstellen f�r die automatische Serialisierung der solcher Daten vorgesehen, welche im Rahmen dieser und einer weiteren Arbeit entwickelt werden sollen.  \cite{Wal14} 

Ziel der Arbeit ist es, Spannungszeitreihen und OPM-Objekte mittels der Metadaten in ein \ac{JSON}-Format zu serialisieren und diese zu \"ubertragen. Allgemein muss jede im Projekt vorkommende Klasse serialisiert werden k\"onnen. 
Die Empf�ngerseite muss letztendlich in der Lage sein, aus den �bertragenen \ac{JSON}-Daten wieder Objekte zu erstellen, welche im Programm weiter verarbeitet werden k�nnen.
Da es sich im Projekt bei allen Klassen um OPM-Objekte handelt, muss es prinzipiell m\"oglich sein, jedes Klassen-Objekt zu serialisieren.

Es soll analysiert werden, welche M�glichkeiten bestehen, um Java-Klassen-Objekte zu serialisieren, beziehungsweise zu deserialisieren. Hierf�r soll ein Vergleich verschiedener Techniken erfolgen, welche zu bewerten sind. 

Die beste M�glichkeit soll im Anschluss der �berpr�fung implementiert und genauer untersucht werden. Hierbei ist ein besonderes Augenmerk auf die Geschwindigkeit des Serialisierungprozesses zu legen. Aber auch grundlegende Aspeke wie Einfachheit der Implementierung, Zusammenspiel mit anderen Serialisierern und die Anwendbarkeit an realit�tsnahen Daten soll �berpr�ft werden.

Es soll des Weiteren gepr�ft werden, ob alle "`Standard-Datentypen"' serialisiert werden k�nnen und wie der Umgang mit Klassen-Attributen, Rekursion und Attributen, die nicht serialisiert werden sollen, beschaffen ist.

Abschlie\ss{}end soll ein Vergleich zeigen, welche Serialisierungsart (JSON oder XML), f�r das Projekt, besser geeignet ist. Hierf�r wird die Arbeit "`Serialisierung von Datenobjekten in XML zur �bertragung von Objekten aus Energieanwendungen"' von Herrn Achim Walz herangezogen. \cite{Wal14}

