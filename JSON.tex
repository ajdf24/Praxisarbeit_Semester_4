\section{JavaScript Object Notation}
\ac{JSON} ist ein Textbasiertes Format zum Datenaustausch, wobei jedes g�ltige JSON-Dokument auch ein g�ltiges JavaScript ist. Es wurde als Ersatz f�r XML geschaffen und wird haupts�chlich in Bereichen eingesetzt wo Ressourcen wie Speicherplatz, Prozessorleistung und Netzwerkverbindung stark limitiert sind. Im Aufbau erinnert \ac{JSON}  an die Struktur eines Arrays. Ein Beispiel f�r ein \ac{JSON}-Objekt ist im Kapitel \ref{Der Aufbau von JSON} zu finden. \cite{WikiJSON}

\subsection{Der Aufbau von JSON}
\label{Der Aufbau von JSON}
Ein Beispiel f�r ein g�ltiges JSON-Dokument ist im Beispiel unten zu finden. In der ersten und letzten Zeile sind geschweifte Klammern zu finden, da jedes JSON-Dokument ein Objekt ist und Objekte in JSON von geschweiften Klammern umschlossen werden m�ssen. \cite{Sai13}

JSON ist nach dem Schl�ssel/Wert Prinzip aufgebaut  was bedeutet, dass jedem Schl�ssel genau ein Wert zugeordnet werden kann. Im Beispiel sind alle in JSON m�glichen Formattypen aufgezeigt.

Zeile zwei enth�lt einen String, eine Zeichenkette in der jedes Zeichen erlaubt ist. Der boolsche Wert wird genau wie ein Nullwert ohne Anf�hrungszeichen geschrieben wie in den Zeilen drei und vier gezeigt.

Zahlen k�nnen Ganzzahlig, Flie�kommazahlen oder Exponentialzahlen sein wie sie auch im Beispiel zu finden sind.

Ein Array kann mehrere Werte enthalten und wird deshalb von eckigen Klammern umschlossen.  Die eigentlichen Werte im Array m�ssen jedoch vom selben Typ sein.

Objekte k�nnen wiederum Objekte enthalten, wie es in den Zeilen acht bis zehn dargestellt ist. Das innere Objekt wird wieder von geschweiften Klammern umschlossen.

Somit k�nnen sechs Datentypen in JSON Unterschieden werden Strings, Zahlen, Booleans, Arrays, Objekte und Nullwerte. Zu beachten ist das Booleans, Nullwerte und Zahlen ohne Anf�hrungszeichen beschrieben werden.

\lstinputlisting{Code/Beispiel.json}

\subsection{JSON in Verbindung mit Programmiersprachen}
Viele Programmiersprachen wie PHP, Python, C\#, C++ und Java unterst�tzen JSON sehr gut und sogar Nativ. 

\subsubsection{JSON und JavaScript}
JSON wird unter JavaScript als ganz normale Variable gef�hrt und kann auch als solche ausgelesen werden. Dies geschieht beispielhaft �ber das Kommando \texttt{alert(JSONVariablenName.Zahl);} liefert den Wert 1234567 aus dem Beispiel, unter der Bedingung das, dass JSON-Objekt als Variable \texttt{JSONVariablenName} deklariert wurde, zur�ck.
