\section{JavaScript Object Notation}
\ac{JSON} ist ein textbasiertes Format zum Datenaustausch, wobei jedes g�ltige JSON-Dokument auch ein g�ltiges JavaScript ist. Jedoch ist \ac{JSON} unabh\"angig von der Programmiersprache. Es wurde als Ersatz f�r XML geschaffen und wird haupts�chlich in Bereichen eingesetzt wo Ressourcen wie Speicherplatz, Prozessorleistung und Netzwerkverbindung stark limitiert sind. Im Aufbau erinnert \ac{JSON}  an die Struktur eines Arrays. Ein Beispiel f�r ein \ac{JSON}-Objekt ist im Kapitel \ref{Der Aufbau von JSON} zu finden. \cite{WikiJSON}

\subsection{Der Aufbau von JSON}
\label{Der Aufbau von JSON}
Ein g�ltiges JSON-Dokument ist im Beispiel unten zu finden. In der ersten und letzten Zeile sind geschweifte Klammern zu finden, da jedes JSON-Dokument ein Objekt, im Sinne von JSON-Objekten, ist und Objekte in JSON von geschweiften Klammern umschlossen werden m�ssen. \cite{Sai13}

JSON ist nach dem Schl�ssel/Wert Prinzip aufgebaut  was bedeutet, dass jedem Schl�ssel genau ein Wert zugeordnet werden kann. Im Beispiel sind alle in JSON m�glichen Formattypen aufgezeigt.

Zeile zwei enth�lt einen String, eine Zeichenkette in der jedes Zeichen erlaubt ist. Der boolsche Wert wird genau wie ein Nullwert ohne Anf�hrungszeichen geschrieben wie in den Zeilen 3 und 4 gezeigt.

Zahlen k�nnen ganzzahlig, Flie�kommazahlen oder Exponentialzahlen sein.

Ein Array kann mehrere Werte enthalten und wird deshalb von eckigen Klammern umschlossen.  Die eigentlichen Werte im Array m�ssen jedoch vom selben Typ sein.

Objekte k�nnen wiederum Objekte enthalten, wie es in den Zeilen acht bis zehn dargestellt ist. Das innere Objekt wird wieder von geschweiften Klammern umschlossen.

Somit k�nnen sechs Datentypen in JSON Unterschieden werden Strings, Zahlen, Booleans, Arrays, Objekte und Nullwerte. Zu beachten ist das Booleans, Nullwerte und Zahlen ohne Anf�hrungszeichen geschrieben werden.

Ein Beipiel f\"ur eine zu serialisierende Klasse, ohne Annotationen f\"ur einen Serialisierer, k\"onnte wie folgt aussehen. Worum es sich bei Annotationen genau handelt wird im Kapitel \ref{Jackson} genauer beschrieben. 
\lstinputlisting{Code/Java_Bsp_f\"ur_JSON.java}

\newpage
Das selbe Beispiel als serilalisiertes JSON-Objekt w\"urde dann wie folgt aussehen k\"onnen.
Je nachdem, mit welcher Biliotheke Serialisiert wird, beziehungsweise welche Annotationen an den Quellcode noch zus\"atzlich angebracht sind, kann das JSON-Objekt auch anders aussehen.
\lstinputlisting{Code/Beispiel.json}

\subsection{JSON in Verbindung mit Programmiersprachen}
Viele Programmiersprachen wie PHP, Python, C\#, C++ und Java unterst�tzen JSON sehr gut und sogar nativ. Dies bedeutet, das f\"ur eine grundlegende Verwendung von JSON keine zus\"atzlichen Bibliotheken ben\"otigt werden. 

Eine Verwendung von JSON ohne einen speziellen Anwendungsfall, der wirklich JSON-Objekte ben\"otigt, wie das Verwenden von Jackson oder MongoDB, ist wenig Sinnvoll. Eine Kapselung von Information in JSON, ist somit nur Sinnvoll wenn auch bestimmte Programmteile daf\"ur ausgelegt sind mit ihnen zu arbeiten.
% das Speichern von Informationen in JSON sonnst umst\"andlich ist und viele Umwandlungsschritte ben\"otigt. Ohne speziellen Anwendungsfall ist die Verwendung von Standartdatentypen in der Regel vorzuziehen.

\subsubsection{JSON und JavaScript}
JSON wird unter JavaScript als ganz normale Variable gef�hrt und kann auch als solche ausgelesen werden. Dies geschieht beispielhaft �ber das JavaScript-Kommando \\\texttt{alert(JSONVariablenName.Zahl)}. Der Aufruf liefert den Wert 1234567 aus dem Beispiel, unter der Bedingung, dass das JSON-Objekt als Variable mit dem Namen \texttt{JSONVariablenName} im JavaScript deklariert wurde, zur�ck.
