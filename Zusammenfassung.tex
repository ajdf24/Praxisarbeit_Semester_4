\section{Zusammenfassung}
Ziel der vorliegenden Arbeit war die Untersuchung der M\"oglichkeiten einer Serialiserung von Java-Instanzen in \ac{JSON}-Dokumente und zur\"uck. Nach der Untersuchung verschiedener Ans\"atze wurde sich, in Zusammenarbeit mit einer anderen Arbeit von Herrn Achim Walz, f\"ur die Verwendung Jackson entschieden. Herr Walz sollte die Serialisierung nach XML  in Zusammenarbeit mit \ac{JAXB} untersuchen. \cite{Wal14}

Nach einer Untersuchung der M\"oglichkeiten, Instanzen in \ac{JSON} zu speichern, wurde sich auf die Analyse und die Umsetzung mit Jackson spezialisiert. Es wurde untersucht, welche Datentypen \ac{JSON} nativ speichern kann und welche M\"oglichkeiten die einzelnen Module der JacksonBibliothek bieten.

Es wurde gezeigt, dass eine vollst\"andige Serialisierung von Java-Instanzen m\"oglich ist, woraufhin auch untersucht wurde, welche M\"oglichkeiten durch Annotations m\"oglich sind. Im Projekt wurden \ac{JAXB}-Annotations verwerdet, welche Herr Achim Walz in dessen Arbeit "`Serialisierung von Datenobjekten in XML zur \"Ubertragung von Objektdaten aus Energieanwendungen"', beschreibt.

Im Laufe der Arbeit wurde ersichtlich, dass eine Spezifikation des \ac{SMD}-Assistenten notwendig wird. Da die Serialisierer auf Daten vom Assistenten zugreift, wurde eine solche erste Spezifikation erstellt und ist nun in der vorliegenden Arbeit zu finden.

\"Ahnlich wie mit dem \ac{SMD}-Assistenten wurde mit dem Klassengenerator verfahren. Da der \ac{CG} die Klassen liefert, welche die Serialisierer f\"ur ihre Arbeit ben\"otigen, wurde auch eine erste Spezifikation vom Klassengenerator erstellt.

Nach der Implementierung der Serialisierer wurden Klassen erstellt und zu Testzwecken serialisiert. Hierbei wurden alle Java-eigenen "`Standard-Datentypen"' getestet. Aber auch das Verhalten bei rekursiven Aufrufen sowie das Einbinden von Klassen-Attributen wurde untersucht.

Beim Serialisieren einer Klasse, welche Massendaten enth\"alt, stellte sich heraus, dass sowohl Jackson als auch JAXB Probleme haben, diese Klasse zu serialisieren. Dies ergab sich durch einen Array-\"Uberlauf in der Klasse \texttt{StringBuffer,} welche von beiden intern aufgerufen wird.

% Bei der Bearbeitung der Aufgabenstellung wurden Konflikte mit dem \ac{OPM}-Modell festgestellt, da Sie Serialisierer unter Java andere Voraussetzungen ben\"otigen als \{OPM} vorgibt.
% 
% worauf einige \"Anderungen im \ac{OPM}-Modell vorgeschlagen wurden, welche zum jetzigen Zeitpunkt zur Diskussion stehen.

Es wurde festgestellt, dass eine Serialiserung mit Jackson Vorteile gegen\"uber JAXB bietet und eine entsprechende Empfehlung gegeben. 

Die Vorteile sind zum einen die gute Einbindung in Java, zum anderen die geringere Gr\"o\ss{}e der serialisierten Instanzen. \ac{JSON} ist im Gegenteil zu XML vollkommen Java-konform und l\"asst sich auch in vielen anderen Programmiersprachen verwenden. 
\section{Ausblick}
Im weiteren Verlauf des Projektes wird nun die Spezifizierung des \ac{SMD}-Assistenten und des ClassGenerators noch einmal \"uberarbeitet und weiter auf dem Gebiet geforscht. 

Der \ac{UDDE}, welcher schon in einer ersten Version existiert, soll nun so erweitert werden, dass er nicht nur einfache Standard-Datentypen verstehen kann, sondern auch Klassen-Attribute sollen sp\"ater durch ihn verarbeitet werden k\"onnen.

Wenn alle Bereiche und Bestandteile des \ac{GDS} spezifiziert sind, wird eine erste Version nach \ac{OPM}-Regeln implementiert.