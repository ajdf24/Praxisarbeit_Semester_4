\section{JavaScript Object Notation}
\ac{JSON} ist ein Textbasiertes Format zum Datenaustausch, wobei jedes g�ltige JSON-Dokument auch ein g�ltiges JavaScript ist. Es wurde als Ersatz f�r XML geschaffen und wird haupts�chlich in Bereichen eingesetzt wo Ressourcen wie Speicherplatz, Prozessorleistung und Netzwerkverbindung stark limitiert sind. \cite{WikiJSON}

\subsection{Der Aufbau von JSON}
Ein Beispiel f�r ein g�ltiges JSON-Dokument ist im Beispiel unten zu finden. In der ersten und letzten Zeile sind geschweifte Klammern zu finden, da jedes JSON-Dokument ein Objekt ist und Objekte in JSON von geschweiften Klammern umschlossen werden m�ssen. \cite{Sai13}

JSON ist nach dem Schl�ssel/Wert Prinzip aufgebaut  was bedeutet, dass jedem Schl�ssel genau ein Wert zugeordnet werden kann. Im Beispiel sind alle in JSON m�glichen Formattypen aufgezeigt.

Zeile zwei enth�lt einen String, eine Zeichenkette in der jedes Zeichen erlaubt ist. Der boolsche Wert wird genau wie ein Nullwert ohne Anf�hrungszeichen geschrieben wie in den Zeilen drei und vier gezeigt.

Zahlen k�nnen Ganzzahlig, Flie�kommazahlen oder Exponentialzahlen sein wie sie auch im Beispiel zu finden sind.
Ein Array kann mehrere Werte enthalten und wird deshalb von eckigen Klammern umschlossen.  Die eigentlichen Werte im Array m�ssen jedoch vom selben Typ sein.

Objekte k�nnen wiederum Objekte enthalten, wie es in den Zeilen acht bis zehn dargestellt ist. Das innere Objekt wird wieder von geschweiften Klammern umschlossen.
\lstinputlisting{Code/Beispiel.json}