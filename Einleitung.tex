\section{Einleitung}
Die folgende Arbeit befasst sich mit der Serialisierung von Datenobjekten in JSON zur �bertragung von Objektdaten. 
\subsection{Generisches Managementsystem f\"ur Energiedaten}
Das \ac{GDS} verwendet Objekte um Anwendungsdaten zu verwalten, die einem \ac{OPM} gen�gen und durch \ac{SMD} beschrieben werden.
Das \ac{GDS} wird zur Zeit im \ac{IAI} des \ac{KIT} entwickelt und ist ein System generischer Datenservices welches bei 
Release vollautomatisch Energiedaten managen kann. 

Durch den generischen Charakter kann es auch Daten aus anderen Bereichen verwalten.
\subsection{Objektorientiertes Datenmodell}
Im \ac{OPM}  werden Richtlinien f�r die Entwicklung von allen objektorientierten Softwarebausteinen festgelegt. Das gesamte Projekt ist bisher OPM-konform gehalten und somit soll auch die Schnittstelle der Serialisierung OPM-konform gestaltet werden.

Im Folgenden sind die wichtigsten OPM-Regeln dargstellt:
\begin{itemize}
 \item Basisklasse (\texttt{OPMObject}) von der alle Klassen Erben
 \item In der Dokumentation eines Attributs wird immer der erlaubte Wertebereich spezifiziert.
 \item Es gibt keine Konstanten
 \item Attribute sind Grunds\"atzlich \texttt{private} und werden durch Getter- und Setter-Methoden aufgerufen
 \item Programme bestehen nur aus Objekten, der Aufruf erfolgt ausschlie\ss{}lich \"Uber Methodenaufrufe
 \item Die Dokumentation muss den Spezifikationen der verwendete Sprache entsprechen
 \item Jede Klasse enth\"alt einen der Stati Valid, Experimental oder Depricated in ihrer Dokumentation
 \item Bezeichner werden im "`Camel Case"' formuliert
\end{itemize}
\subsection{Strukturelle Metadaten}
Strukturelle Metadaten sind laut der \ac{OPM} Definition spezielle Metadaten, die den Aufbau einer Programmklasse enthalten. Die Informationen der \ac{SMD} sind programmiersprachenunabh�ngig und stellen sie den Anwendungsprogrammen zur Verf�gung.

Die Metadaten werden in einer SQL-Datenbank gespeichert, wo sie wenn ben�tigt geladen werden k�nnen.
